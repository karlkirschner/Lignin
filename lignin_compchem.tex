\documentclass[10pt,a4paper]{article}
\usepackage[utf8]{inputenc}
\usepackage{amsmath}
\usepackage{amsfonts}
\usepackage{amssymb}
\author{Karl Kirschner}
\title{Lignin CompChem Aspect}
\begin{document}



%hydroxyphenyl (H unit)
%guaiacyl (G unit)
%syringyl (S unit),

%Source: ZhangHLWHFALZ2019
%The lignin structure contains various functional groups, including methoxy, phenolic hydroxyl, aliphatic hydroxyl, carboxyl and carbonyl groups.

\textbf{Lignin-related computational chemistry literature}
\begin{itemize}
    \item Reviews: \cite{PetridisPUHOFRS2011,SanghaPSZP2011,MurilloBNM2017,MurilloBNM2017,ZhangHLWHFALZ2019}
    
    \item Quantum mechanics: \cite{MardisGBPG1999,DurbeejE2003a, DurbeejE2003b, BesteB2009, ChenS2010, JarvisDCDSDRN2011, KimCNBFPB2011, RodrigoJZ2011, WattsMK2011, QinWZDY2014, HuangHLTWW2015, HuangH2015, ChenYLSLFWC2015, ElderF2016, JunjiaoJCTCW2016, Sanchez-GonzalezMD2017, ZhangHDFZ2017, GaniOASBBR2018}

    \item Force field development: \cite{PetridisS2009,VermaasPRCB2019}
    
    \item Molecular dynamics: \cite{BesombesM2005, LiE2005, BesombesM2005a, PetridisSS2011, CharlierM2012, Castilho-AlmeidaDD2013, LindnerPSS2013, LanganPOPFNSLHHHUEGRS2014, LiWXZ2014, VermaasPQSLS2015, PetridisS2016, JunjiaoJCTCW2016}

    \item ReaxFF: \cite{Beste2014}

    \item Coarse-grained: \cite{LiZHZZ2016}


\end{itemize}







\bibliographystyle{unsrt}
\bibliography{lignin}
\end{document}